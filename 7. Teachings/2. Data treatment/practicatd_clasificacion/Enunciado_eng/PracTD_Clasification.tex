\documentclass[11pt,a4paper]{article}

\usepackage{epsfig}
\usepackage[OT1]{fontenc}
\usepackage[latin1]{inputenc}
\usepackage[spanish]{babel}
\usepackage{amsmath,amssymb}
\usepackage{float}

\oddsidemargin  -0.31in
\evensidemargin -0.31in

\topmargin 50truept
\headheight 0truept
\headsep 0truept
\voffset -20truemm
%\footheight 0truept
\footskip 10truemm
\textheight 250truemm
\textwidth 175truemm

%\twocolumn
%\columnsep 8truemm
\pagestyle{empty}

\setlength{\parindent}{0pt}
\setlength{\parskip}{10pt}


\def\bfmath#1{\mbox{\boldmath$#1$}} 
\begin{document}

%%%%%%%%%%%%%%%%%%%%%%%%%%%%%%%%%
\title{Pr�ctica 1: Classification}

%%% IEEE
\author{Data Processing \\ Academic year 2013--2014 }
\maketitle

\section*{Introduction}
\label{sec-intro}

In this lab session we will study logistic regression and support vector machines using artificially generated data. After that, we will analyze the behavior of these classifiers using a real dataset.


%%%%%%%%%%%%%%%%%%%%%%%%%%%%%%%%%%%%%%%%%%%%%%%%%%%%%%%%%
\section{Analysis of a bidimensional synthetic problem}

Load the data contained in file {\tt 'datosP1.mat'}. This file contains two variables, {\tt x} and {\tt y}. The first one is an observation  matrix. The second one contains the class label (0 or 1) corresponding to each observation.



%%%%%%%%%%%%%%%%%%%%%%%%%%%%%%%%%%%%%%%%%%%%%%%%%%%
\subsection{Logistic regression. Linear classifier}

First, we will analyze the behavior of a logistic regression model.

\begin{enumerate}
\item Visualize the observations on the plane, using a different marker for the data from each class.
\item Using function {\tt glmfit}, fit a logistic regression model to the data.
\item Using function {\tt contourf}, visualize the posterior probability map for class 1.
\item Determine the training and validation error rates.
\item Determine the training and validation data likelihoods for the estimated model.
\end{enumerate}


%%%%%%%%%%%%%%%%%%%%%%%%%%%%%%%%%%%%%%%%%%%%%%%%%%%%%%%%
\subsection{Logistic regression. Nonlinear classifier}

In the following, we will analyze the behavior of a non-lineal classifier based in a polynomial logistic regression model.

\begin{enumerate}
\item Using function {\tt glmfit}, fit a logistic regression model with polynomial terms up to degrees 1, 2 and 3.
\item Using {\tt contourf}, visualize the estimated posterior probability map for class 1.
\item Compute the training and validation error rates.
\item Determine the training and validation data likelihoods for the estimated model.
\end{enumerate}


%%%%%%%%%%%%%%%%%%%%%%%%%%%%%%%%%%%%%%%
\subsection{M�quina de vectores soporte}

We will evaluate the performance of a classifier based on Support Vector Machines (SVM).

We will use an SVM with Gaussian kernels, taking kernel width $\sigma=0.5$.

\begin{enumerate}
\item Using function {\tt svmtrain}, train an SVM with the training data, visualizing the decision boundary and taking $C=$ 0.1.
\item Exploring different values of parameter $C$, visualize the training and validation error rates, as a function of $C$, and choose a specific value according to the validation error.
\end{enumerate}


%%%%%%%%%%%%%%%%%%%%%%%%%%%%%
\subsection{Final evaluation}

Choose, according to the validation error, the classifier providing the best performance, among all those analyzed in the sections above, and compute the test error rate.


%%%%%%%%%%%%%%%%%%%%%%%%%%%%%%%%%%%%%%%%
\section{Classification with real data}

We will apply logistic regression an SVM to a real multidimensional dataset. The data visualization in the input space is no longer possible, but performance evaluation can be carried out using the error rates in any case.

We will work with the {\tt cancer\_dataset}.


%%%%%%%%%%%%%%%%%%%%%%%%%%%%%
\subsection{Data preparation}


In order to ensure that all variables have a similar scale, data normalization is generally advised.

The following code fragments carries a linear transformation making the observations in the training set have zero mean and unit sample variance. The same transformation must be applied to validation and test data:

{\tt mx = mean(xTrain); stdx = std(xTrain);}  \\
{\tt xTrain = (xTrain - ones(nTrain,1)*mx)./(ones(nTrain, 1)*stdx);} \\ 
{\tt xVal   = (xVal   - ones(nVal,1)*mx)./(ones(nVal,1)*stdx); }     \\
{\tt xTest  = (xTest  - ones(nTest,1)*mx)./(ones(nTest,1)*stdx);}

We will use the normalized data all along the rest of the lab exercise.


%%%%%%%%%%%%%%%%%%%%%%%%%%%%%%%%%%%%%%%%%%%%%%%%%%%%%%%
\subsection{Classification with Support Vector Machines}

In this section we will train a SVM with Gaussian kernels. We will use a validation dataset to determine the values of the hyperparameters $\sigma$ and $C$.

\begin{enumerate}
\item Train the SVM for different values of $\sigma$ ranging from -0.02 to 8 and values of $C$ between 0.001 and 1000. Compute the training and validation eror rates in each case, and represent graphically (e.g., by means of function {\tt contourf}) such error rates.
\item Compute the values of the kernel width, $\sigma$, and parameter $C$ minimizing the validation error rate.
\item Evaluate the classifier performance using the test data, for the selected hyperparameter values.
\end{enumerate}





\end{document}


